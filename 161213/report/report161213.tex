\documentclass[12pt,a4paper,fleqn]{article}
\title{Progress Report}
\author{Syed Ahmad Raza}
\date{2016.12.13}
\usepackage{mathtools}
\usepackage{graphicx}
\usepackage{color}          % for color eps output

\begin{document}
\maketitle \section*{Numerical differentiation using arrays for the first
derivative}
Arrays were utilized in an algorithm coded in C++ for numerical differentiation
using three different commonly used schemes. The function sin$(3x)$ was
differentiated using forward difference, backward difference and central
difference methods.

\subsection*{Forward difference method}
\begin{equation}
f^\prime(x) = \frac{f(x_{i+1})-f(x_i)}{x_{i+1}-x_i}
\end{equation}

\subsection*{Backward difference method}
\begin{equation}
f^\prime(x) = \frac{f(x_i)-f(x_{i-1})}{x_i-x_{i-1}}
\end{equation}

\subsection*{Central difference method}
\begin{equation}
f^\prime(x) = \frac{f(x_{i+1})-f(x_{i-1})}{x_{i+1}-x_{i-1}}
\end{equation}

The results are shown in the figures below.

\begin{figure}[p!]
\centering
\input{firstsin10}
\caption{Comparison of numerical differentiation of sin$(3x)$ using 10 intervals
for the first derivative}
\end{figure}
\begin{figure}[p!]
\centering
\input{firstsin100}
\caption{Comparison of numerical differentiation of sin$(3x)$ using 100
intervals for the first derivative}
\end{figure}

\section*{Numerical differentiation using arrays for the second
derivative}
C++ arrays were also used for numerical differentation of the second
derivative using schemes similar to those for first derivative. The function
sin$(3x)$ was differentiated twice using forward difference, backward difference
and central difference methods.

\subsection*{Forward difference method for second derivative}
\begin{equation}
f''(x) = \frac{f(x_{i+2})-2f(x_{i+1})+f(x_i)}{h^2}
\end{equation}

\subsection*{Backward difference method for second derivative}
\begin{equation}
f''(x) = \frac{f(x_i)-2f(x_{i-1})+f(x_{i-2})}{h^2}
\end{equation}

\subsection*{Central difference method for second derivative}
\begin{equation}
f''(x) = \frac{f(x_{i+1})-2f(x)+f(x_{i-1})}{h^2}
\end{equation}

The results for the second derivative are shown in the figures below.
\begin{figure}[p!]
\centering
\input{secondsin10}
\caption{Comparison of numerical differentiation of second derivative
of sin$(3x)$ using 10 intervals}
\end{figure}
\begin{figure}[p!]
\centering
\input{secondsin100}
\caption{Comparison of numerical differentiation of second derivative
of sin$(3x)$ using 100 intervals}
\end{figure}

\newpage
\section*{Determination of error for for first and second derivatives}
The error for calculation of the first derivative was determined for
each of the three differnt methods using the following formula:
\begin{equation}
E = \sqrt{\frac{\sum\limits_{i=1}^{N}\left(\frac{A_N-A_a}{A_a}\right)^2}{N}}
\end{equation}
The calculations were performed for intervals starting from 10 to 10,000 over
the range 0 to $pi$. The plots are shown below.
\begin{figure}[p!]
\centering
\input{firstsinfwderror}
\caption{Plot of log\textsubscript{10}$E$ versus log\textsubscript{10}$N$ for
numerical differentiation of first derivative using forward difference method}
\end{figure}
\begin{figure}[p!]
\centering
\input{firstsinbwderror}
\caption{Plot of log\textsubscript{10}$E$ versus log\textsubscript{10}$N$ for
numerical differentiation of first derivative using backward difference method}
\end{figure}
\begin{figure}[p!]
\centering
\input{firstsincnterror}
\caption{Plot of log\textsubscript{10}$E$ versus log\textsubscript{10}$N$ for
numerical differentiation of first derivative using central difference method}
\end{figure}

The error for calculation of the second derivative was also determined for
each of the three differnt methods using the same formula. The plot are as
follows.
\begin{figure}[p!]
\centering
\input{secondsinfwderror}
\caption{Plot of log\textsubscript{10}$E$ versus log\textsubscript{10}$N$ for
numerical differentiation of second derivative using forward difference method}
\end{figure}
\begin{figure}[p!]
\centering
\input{secondsinbwderror}
\caption{Plot of log\textsubscript{10}$E$ versus log\textsubscript{10}$N$ for
numerical differentiation of second derivative using backward difference method}
\end{figure}
\begin{figure}[p!]
\centering
\input{secondsincnterror}
\caption{Plot of log\textsubscript{10}$E$ versus log\textsubscript{10}$N$ for
numerical differentiation of second derivative using central difference method}
\end{figure}
\newpage
\section*{Numerical solution for 1D heat transfer}
Forward time center in space (FTCS) method was used with the following formula:
\begin{equation}
T_i^{n+1} = T_i^n + \Delta t(\alpha \frac{\partial^2T}{\partial x^2})
\end{equation}
\end{document}