\documentclass[12pt,a4paper,fleqn]{article}
\title{Progress Report}
\author{Syed Ahmad Raza}
\date{2017.05.16}
\usepackage{mathtools}
\usepackage{graphicx}
\usepackage{color}          % for color eps output
\usepackage{afterpage}
\usepackage{newtxtext}
\usepackage{newtxmath}
%\usepackage{layouts}       % for: \printinunitsof{in}\prntlen{\textwidth}

\begin{document}
	\maketitle
	\section*{\raggedright Solving the pressure term for solution of Navier-Stokes equations using Finite Volume Method}
    
    Continuity equation:
    \begin{equation} \label{eq:continuity}
    \nabla \cdot \mathbf{u} = 0
    \end{equation}
    Navier-Stokes equation:
    \begin{equation}\label{eq:navier-stokes}
    \frac {\partial \mathbf{u}}{\partial t}+\nabla \cdot (\mathbf{u}\mathbf{u}) = -\frac{1}{\rho}\nabla p + \nu \nabla^2 \mathbf{u}+\mathbf{f} \quad .
    \end{equation}

    \subsection*{\raggedright Projection method}
    Ignoring the body force, projection method is used to decompose Navier-Stokes equation into
    \begin{eqnarray}
    \frac{\mathbf{u}^*-\mathbf{u}^n}{\Delta t} = -\nabla \cdot (\mathbf{u}\mathbf{u})+ \nu \nabla^2 \mathbf{u}
    \label{eq:projection01}\\
    \frac{\mathbf{u}^{n+1}-\mathbf{u}^*}{\Delta t}=-\frac{1}{\rho}\nabla p
    \label{projection02}
    \end{eqnarray}
    where {\it n} is the time level index.
    
    \subsection*{\raggedright Poisson equation of pressure}
    The pressure term can be solved using
    \begin{equation} \label{eq:poisson}
    \nabla^2 p = \frac{\rho}{\Delta t}\nabla \cdot \mathbf{u}^* \quad ,
    \end{equation}
    which can be represented as
    \begin{equation} \label{eq:poisson-componenets}
    \frac{\partial^2 p}{\partial x^2} + \frac{\partial^2 p}{\partial y^2}
    = \frac{1}{\Delta t} \left(\frac{\partial u^*}{\partial x} + \frac{\partial v^*}{\partial y}\right) \quad .
    \end{equation}
    Integrating it twice, discretizing and rearranging leads to
    \begin{align}
    p_{i,j} =
    &\frac{1}{\left[ - \frac{Dy}{Dx1} - \frac{Dy}{Dx2} - \frac{Dx}{Dy1} - \frac{Dx}{Dy2} \right]}
    \nonumber \\
    &\times
    \left[
    - \frac{Dy}{Dx1}p_{i+1,j} - \frac{Dy}{Dx2}p_{i-1,j} - \frac{Dx}{Dy1}p_{i,j+1} - \frac{Dx}{Dy2}p_{i,j-1}
    \right. \nonumber \\
    &\left.
    + \frac{1}{\Delta t}\left\{
    \left(u^*_{i,j}-u^*_{i-1,j}\right) \Delta y
    + \left(v^*_{i,j}-v^*_{i,j-1}\right) \Delta x
    \right\}
    \right]
    \quad .
    \end{align}
    Using successive over-relaxation method (SOR),
    \begin{align}
    p_{i,j} =& \left(1 - \omega\right)p_{i,j} + \omega
    \Bigg[
    \frac{1}{\left( - \frac{Dy}{Dx1} - \frac{Dy}{Dx2} - \frac{Dx}{Dy1} - \frac{Dx}{Dy2} \right)}
    \nonumber \\
    &\times
    \left\{
    - \frac{Dy}{Dx1}p_{i+1,j} - \frac{Dy}{Dx2}p_{i-1,j} - \frac{Dx}{Dy1}p_{i,j+1} - \frac{Dx}{Dy2}p_{i,j-1}
    \right. \nonumber \\
    &\left.
    + \frac{1}{\Delta t}\left(
    \big(u^*_{i,j}-u^*_{i-1,j}\big) \Delta y
    + \big(v^*_{i,j}-v^*_{i,j-1}\big) \Delta x
    \right)
    \right\}
    \Bigg]
    \quad ,
    \end{align}
    where the relaxation factor, $\omega = 1.8$.
    
    Finally, the correct velocity can be found using
    \begin{equation}
    \mathbf{u}^{n+1} = \mathbf{u}^* - \frac{\Delta t}{\rho}\cdot \nabla p \quad ,
    \end{equation}
    which is, for $u$- and $v$-components,
    \begin{equation}
    u^{n+1}_{i,j} = u^*_{i,j} - \frac{\Delta t}{\rho}\cdot \frac{p_{i+1,j} - p_{i,j}}{\Delta x}
    \end{equation}
    and
    \begin{equation}
    v^{n+1}_{i,j} = v^*_{i,j} - \frac{\Delta t}{\rho}\cdot \frac{p_{i,j+1} - p_{i,j}}{\Delta y}
    \end{equation}
    
    
    
\end{document}
%\subsection*{\raggedright Euler scheme for first time step}
%For the first time step, the Euler scheme is adopted for the intermediate velocity using
%\begin{align}
%\frac{u^*_{i,j}-u^{}_{i,j}}{\Delta t} =
%{}& - \frac{u_e^2 - u_w^2}{\Delta x} - \frac{u_n v_n - u_s v_s}{\Delta y}
%\nonumber \\
%& + \nu\left[
%\left\{
%\frac{u_{i+1,j}-u_{i,j}}{x_{i+2}-x_{i+1}}
%- \frac{u_{i,j}-u_{i-1,j}}{x_{i+1}-x_i}
%\right\}
%\frac{1}{\Delta x}
%\right.
%\nonumber\\
%& \left. + \left\{
%\frac{u_{i,j+1}-u_{i,j}}{(y_{j+2}-y_j)/2}
%- \frac{u_{i,j}-u_{i,j-1}}{(y_{j+1}-y_{j-1})/2}
%\right\}
%\frac{1}{\Delta y}
%\right] \quad ,
%\label{eq:Euler-u}
%\end{align}
%where
%\begin{equation*}
%\begin{aligned}
%u_e &= u_{i,j}\\
%u_w &= u_{i-1,j}\\
%u_n &= u_{i,j} + \frac{u_{i,j+1} - u_{i,j}}{(y_{j+2}-y_j)/2}\\
%u_s &= u_{i,j} - \frac{u_{i,j} - u_{i,j-1}}{(y_{j+1}-y_{j-1})/2}
%\end{aligned}
%\qquad\qquad
%\begin{aligned}
%v_n &= v_{i,j}\\
%v_s &= v_{i,j-1}\quad .\\
%{}\\
%{}\\
%{}
%\end{aligned}
%\end{equation*}
%and
%\begin{align}
%\frac{v^*_{i,j}-v^{}_{i,j}}{\Delta t} = 
%{}& - \frac{u_e v_e - u_w v_w}{\Delta x} - \frac{v_n^2 - v_s^2}{\Delta y} \nonumber \\
%& + \nu\left[
%\left\{
%\frac{v_{i+1,j}-v_{i,j}}{x_{i+2}-x_{i+1}}
%- \frac{v_{i,j}-v_{i-1,j}}{x_{i+1}-x_i}
%\right\}
%\frac{1}{\Delta x}
%\right.\nonumber\\
%& \left. + \left\{
%\frac{v_{i,j+1}-v_{i,j}}{(y_{j+2}-y_{j+1})/2}
%- \frac{v_{i,j}-v_{i,j-1}}{(y_{j+1}-y_{j-1})/2}
%\right\}
%\frac{1}{\Delta y}
%\right] \quad .
%\label{eq:Euler-v}
%\end{align}
%where
%\begin{equation*}
%\begin{aligned}
%v_n &= v_{i,j}\\
%v_s &= v_{i,j-1}\\
%v_e &= v_{i,j} + \frac{v_{i+1,j} - v_{i,j}}{(x_{i+2}-x_i)/2}\\
%v_w &= v_{i,j} - \frac{v_{i,j} - v_{i-1,j}}{(x_{i+1}-x_{i-1})/2}
%\end{aligned}
%\qquad\qquad
%\begin{aligned}
%u_e &= u_{i,j}\\
%u_w &= u_{i-1,j}\quad .\\
%{}\\
%{}\\
%{}
%\end{aligned}
%\end{equation*}
%
%\subsection*{\raggedright Adams-Bashforth scheme}
%Denoting the terms on the right-hand side of equation \eqref{eq:projection01} with $\mathcal{F}(\mathbf{u})$, the second order Adams-Bashforth scheme can be applied using
%\begin{equation}\label{eq:adams-bashforth}
%\frac{\mathbf{u}^*-\mathbf{u}^n}{\Delta t} = \frac{3}{2}\mathcal{F}(\mathbf{u}^n)-\frac{1}{2}\mathcal{F}(\mathbf{u}^{n-1})\quad .
%\end{equation}
%
%Applying this formula to a 2D problem, the discretized equations can be written as
%\begin{align}
%\mathcal{F}(u_{i,j}^n) = {}& - \frac{u_e^2 - u_w^2}{\Delta x} - \frac{u_n v_n - u_s v_s}{\Delta y} + \nu\left[
%\left\{
%\frac{u_{i+1,j}-u_{i,j}}{x_{i+2}-x_{i+1}}
%- \frac{u_{i,j}-u_{i-1,j}}{x_{i+1}-x_i}
%\right\}
%\frac{1}{\Delta x}
%\right.\nonumber\\
%& \left. + \left\{
%\frac{u_{i,j+1}-u_{i,j}}{(y_{j+2}-y_j)/2}
%- \frac{u_{i,j}-u_{i,j-1}}{(y_{j+1}-y_{j-1})/2}
%\right\}
%\frac{1}{\Delta y}
%\right] \quad ,
%\label{eq:Fu}
%\end{align}
%and
%\begin{align}
%\mathcal{F}(v_{i,j}^n) = {}& - \frac{u_e v_e - u_w v_w}{\Delta x} - \frac{v_n^2 - v_s^2}{\Delta y} + \nu\left[
%\left\{
%\frac{v_{i+1,j}-v_{i,j}}{x_{i+2}-x_{i+1}}
%- \frac{v_{i,j}-v_{i-1,j}}{x_{i+1}-x_i}
%\right\}
%\frac{1}{\Delta x}
%\right.\nonumber\\
%& \left. + \left\{
%\frac{v_{i,j+1}-v_{i,j}}{(y_{j+2}-y_{j+1})/2}
%- \frac{v_{i,j}-v_{i,j-1}}{(y_{j+1}-y_{j-1})/2}
%\right\}
%\frac{1}{\Delta y}
%\right] \quad .
%\label{eq:Fv}
%\end{align}
%Then, the intermediate velocity will be
%\begin{eqnarray}
%\frac{u^*_{i,j} - u^{}_{i,j}}{\Delta t} = \frac{3}{2}\mathcal{F}(u_{i,j}^{n+1}) - \frac{1}{2}\mathcal{F}(u_{i,j}^{n}) \\
%\frac{v^*_{i,j} - v^{}_{i,j}}{\Delta t} = \frac{3}{2}\mathcal{F}(v_{i,j}^{n+1}) - \frac{1}{2}\mathcal{F}(v_{i,j}^{n})
%\end{eqnarray}
